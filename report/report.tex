\documentclass[a4paper]{article}

\usepackage[UKenglish]{babel}
\usepackage{graphicx}
\usepackage{amsmath}
\usepackage{amssymb}
\usepackage[T1]{fontenc}
\usepackage{listings}

\AtBeginDocument{\renewcommand{\abstractname}{Abstract}}

\begin{document}
\begin{titlepage}
	\begin{center}
	\textsc{\LARGE Machine Learning in Practice\\}
	\textsc{\Large Report}\\[1.5cm]
	\includegraphics[height=100pt]{logo}
   
	\vspace{0.4cm}
	\textsc{\Large Radboud University Nijmegen}\\[.5cm]
	\hrule
	\vspace{0.4cm}
	\textbf{\huge Acquire Valued Shoppers Challenge}\\[0.4cm]
	\hrule
	\vspace{2cm}
	\begin{minipage}[t]{0.45\textwidth}
	\begin{flushleft} \large
	Wouter Geraerdts\\
	sXXXXXX\\[0.7cm]
	Matthijs Hendriks\\
	s4068459\\[0.7cm]
	\end{flushleft}
	\end{minipage}
	\begin{minipage}[t]{0.45\textwidth}
	\begin{flushright} \large
	Thomas N\"agele\\
	s4031253\\[0.7cm]
	Mathijs Vos\\
	s4024443\\[0.7cm]
	\end{flushright}
	\end{minipage}
	\vspace{.7cm}
	
	\begin{abstract}
		ABSTRACT HIERZO
	\end{abstract}
	\vspace{.7cm}

	{\large 3 July 2014}
	\vfill
	\end{center}

\end{titlepage}

\newpage

\section{Introduction}
For this final assignment for Machine Learning in Practice, we chose the \emph{Acquire Valued Shoppers Challenge} from Kaggle\footnote{https://www.kaggle.com/c/acquire-valued-shoppers-challenge}. The goal for this competition is to find out which shoppers will become regular buyers of a certain product after receiving a personalized coupon offer. This data is useful for stores so they can make a relavant offer to their clients at the checkout.

The dataset contains a few tables which contain at least one year of transactions for each client before making an offer to them. It is therefore possible to view a complete buyers history of one client based on his or her transactions. There are also trainHistory file which contain information about an offer as presented to a customer including the outcome: whether this client has become a regular buyer or not. An offers file gives you more information about the offers: the company, department etc. are stored there. For every customer there is at least one year of transactions given, recorded before an offer was made. Transaction information includes a product category, company, department and brand as well a information about the product that is bought, such as the size, quantity and amount.

We have chosen this competition because we wanted to experiment what it was like to work with a dataset of this size and because it also has some very common real-life applications. The challenge of processing 22 GB of data, which doesn't fit in our RAM anymore would be the biggest. It is also a nice idea that there is quite some money to win in this competition.

\section{Approach}


\section{Features}


\section{Classification}


\section{Results}


\section{Conclusions}


\end{document}
